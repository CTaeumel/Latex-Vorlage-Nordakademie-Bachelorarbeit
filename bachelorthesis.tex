\documentclass[titel]{naklatex}

\titleformat{\section}{\normalfont\Large\bfseries}{\thesection}{1em}{}[{\titlerule[0.8pt]}]

% Blindtext für die Vorlage
\usepackage{blindtext}

% Zitationseinstellungen
% \usepackage[backend=biber, style=bath]{biblatex}
\usepackage[backend=biber, style=apa]{biblatex}
\ExecuteBibliographyOptions{sorting=nty}
\addbibresource{quellen.bib}

% Akronyme/Abkürzungen und Glossareinträge importieren
\loadglsentries{dokumente/0_akronyme.tex}
\loadglsentries{dokumente/0_glossareintraege.tex}

% Anhangsverzeichnis
\newcommand{\anhangsverzeichnis}{Anhangsverzeichnis}
\newlistof{anhangselement}{ahv}{\anhangsverzeichnis}
\newcommand{\anhangselement}[1]
{
    \refstepcounter{anhangselement}
    \addcontentsline{ahv}{anhangselement}
    {\protect\numberline{\theanhangselement}#1}\par
}

% Dokument zusammenbauen
\begin{document}
	\begin{frontmatter}
		\thispagestyle{empty}

% Design-Whitespace
\vspace{1.5cm}

% Logo
\begin{figure}[h]
    \centering
    \includegraphics[width=0.9\textwidth]{dokumente/bilder/Nordakademie_Logo_gross.jpg}
\end{figure}

% Design-Whitespace
\vspace{2.5cm}

% Titel und Untertitel
\begin{center}
    \textbf{\Huge{Ein toller Bachelorarbeitstitel}}

    \textbf{\Large{Und ein noch viel besserer und längerer Untertitel, den diese Arbeit definitiv braucht (oder auch nicht)}}
\end{center}

% Design-Whitespace
\vspace{1.5cm}

% Akademischer Grad und Studiengang
\begin{center}
    zur Erlangung des akademischen Grades\\
    \textbf{\large{Bachelor of Science}}
\end{center}

% Design-Whitespace
\vspace{4.0cm}

% Metadaten der Arbeit
\begin{flushleft}
    \begin{tabular}{llll}
        \textbf{Autor:} & & Vorname Nachname & \\
        & & Straße Hausnummer & \\
        & & Postleitzahl Ort & \\
        & & Email: Email & \\
        & & \\
        \textbf{Studiengang:} & & Angewandte Informatik & \\
        & & Matrikelnummer: XXXX & \\ 
        & & \\
        \textbf{Abgabedatum:} & & \today &\\
        & & \\
        \textbf{Erstgutachter:} & & Prof. Dr. X &\\
        \textbf{Zweitgutachter:} & & Prof. Dr. Y &\\
        \textbf{Betrieblicher Betreuer:} & & Prof. Dr. Z &\\
    \end{tabular}
\end{flushleft}

\pagebreak
		\sectionbreak
		% Abstract
		\inhaltsverzeichnis
		\abbildungsverzeichnis
		\tabellenverzeichnis
		\clearpage
		\phantomsection
		\printglossary[type=\acronymtype]
		\clearpage
		\phantomsection
		\printglossary
	\end{frontmatter}

	\begin{mainmatter}
		\section{Unsere Obst und Backwarenwelt}
\label{sec:ObstUndBackwaren}

Unsere Küche erfreut sich vieler Gerichte, unter anderem leckeren \Gls{kuchen} und \Glspl{muffin}. Allerdings spielen auch \Glspl{banane} eine wichtige Rolle (\cite{halimSonificationNovelApproach2006}).

Davon abgesehen gibt es aber auch andere schöne Dinge in dieser Welt wie z.B. Sonifkation (\cite{hermannListenYourData1999}). Dabei handelt es sich um ein tolles Konzept, das unter anderem von \citeauthor{hermannSonificationHandbook2011a} untersucht wurde.

Im Verkehr spielen vor Allem \Glspl{pkw} und \Glspl{lkw} eine Rolle. Diese werden u.a. von der \Gls{basf} produziert (\cite{erdmannSonifikationBildern}).

\subsection{Weitere Beispiele}
\blindtext

\begin{table}[ht]
    \begin{tabular}{ c c c }
        cell1 & cell2 & cell3 \\ 
        cell4 & cell5 & cell6 \\  
        cell7 & cell8 & cell9 \\
    \end{tabular}
    \caption{Übersicht aller möglichen Zellenbeschriftungen}
\end{table}

\subsection{Freudsche Bananen}
\blindtext

\begin{figure}[ht]
    \includegraphics[width=8cm]{dokumente/bilder/Nordakademie_Logo_gross.jpg}
    \centering
    \caption{Logo der Nordakademie}
\end{figure}

\blindtext
	\end{mainmatter}

	\begin{backmatter}
		\literaturverzeichnis
		\section*{Anhang}
\addcontentsline{toc}{section}{Anhang}

% Anhangsverzeichnis einfügen
\listofanhangselement

Anhänge sind toll, hier sind ein paar:

\subsection*{Mehr Bilder von Kartoffeln}
\anhangselement{Mehr Bilder von Kartoffeln}
\blindtext

\subsubsection*{Mehr Bilder von runden Kartoffeln}
\anhangselement{Mehr Bilder von runden Kartoffeln}
\blindtext

\subsubsection*{Mehr Bilder von eckigen Kartoffeln}
\anhangselement{Mehr Bilder von eckigen Kartoffeln}
\blindtext

\subsection*{Mehr Bilder von Tomaten}
\anhangselement{Mehr Bilder von Tomaten}
\blindtext
		\section*{Eidesstaatliche Erklärung}
\addcontentsline{toc}{section}{Eidesstaatliche Erklärung}

Ich erkläre hiermit an Eides statt, dass ich die vorliegende Arbeit selbständig verfasst und dabei keine anderen als die angegebenen Hilfsmittel benutzt habe. Sämtliche Stellen der Arbeit, die im Wortlaut oder dem Sinn nach Publikationen oder Vorträgen anderer Autoren entnommen sind, habe ich als solche kenntlich gemacht. Die Arbeit wurde bisher weder gesamt noch in Teilen einer anderen Prüfungsbehörde vorgelegt und auch noch nicht veröffentlicht.

\vspace{1.5cm}

\begin{flushright}
    Ort, den \today

    \vspace{0.7cm}
    Vorname Nachname
\end{flushright}
	\end{backmatter}
\end{document}